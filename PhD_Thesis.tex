% Template v1.0
% By William Diego <william.diego@outlook.com>
% Version 1.0 released 06/08/2015
%-------------------------- Start ---------------------
\documentclass[twoside,12pt]{thesis_tb}
%\documentclass[11pt]{book}

%%%=============================================================
% Draft Activation: notes
%%%=============================================================
\usepackage{ifdraft, blindtext}
\ifdraft{
    \usepackage{todonotes}}{ % draft with TODOs
    \usepackage[disable]{todonotes} % final without TODOs
}
%%%=============================================================
\usepackage{enumerate}
\usepackage{fancyhdr}
\usepackage[printonlyused,withpage]{acronym}
\usepackage{graphics}
\usepackage{graphicx}
\usepackage{hypernat} %%%% Activate link blue
\usepackage{lmodern}%font modern
\usepackage[titletoc,title]{appendix} %%% Appendix package
\usepackage{bibunits}
\usepackage{cite}
\usepackage{fancybox}
\usepackage[leftcaption]{sidecap}
\usepackage[labelsep=endash, textfont={footnotesize, singlespacing}, margin=10pt, format=plain, labelfont=bf]{caption}
\usepackage{fncychap} %en tete chapitrage
\usepackage{setspace} %interligne simple, double etc...
\usepackage{fancybox}
\usepackage{chngcntr}
\setcounter{secnumdepth}{5}
\setcounter{tocdepth}{3}
\usepackage{caption} 
\usepackage{subcaption}
\renewcommand\theparagraph{\alph{paragraph})}
\RequirePackage[T1]{fontenc}
\RequirePackage[utf8]{inputenc}
\usepackage[frenchb,english]{babel}
\usepackage[pdftex]{hyperref}	
\usepackage{aeguill}
\usepackage{times}
\usepackage{listingsutf8}
\usepackage{makeidx}
\usepackage{glossaries}
\usepackage{hyperref}
\usepackage{amsmath}
\usepackage{amssymb}
\usepackage{tabularx}
\usepackage{epsfig, floatflt,amssymb} 
\usepackage{moreverb}
\usepackage{multirow}
\usepackage{url}
\usepackage[all]{xy}
\usepackage{textcomp}
\usepackage[right]{eurosym}
\usepackage{algorithm, algorithmic}
\usepackage{epstopdf}
\usepackage{lscape}
%%%=============================================================
% re-definition of Chapter Title
%%%=============================================================
\makeatletter
\ChNameVar{\fontsize{14}{16}\usefont{OT1}{phv}{m}{n}\selectfont}
\ChNumVar{\fontsize{60}{62}\usefont{OT1}{ptm}{m}{n}\selectfont}
\ChTitleVar{\Huge\bfseries\rm}
\ChRuleWidth{2pt} % Set RW=4pt
%======
\ChNameUpperCase % Make name uppercase
\ChTitleUpperCase

  \renewcommand{\DOCH}{%
    \settowidth{\px}{\CNV\FmN{\@chapapp}}
    \addtolength{\px}{2pt}
    \settoheight{\py}{\CNV\FmN{\@chapapp}}
    \addtolength{\py}{1pt}

    \settowidth{\mylen}{\CNV\FmN{\@chapapp}\space\CNoV\thechapter}
    \addtolength{\mylen}{1pt}
    \settowidth{\pxx}{\CNoV\thechapter}
    \addtolength{\pxx}{-1pt}

    \settoheight{\pyy}{\CNoV\thechapter}
    \addtolength{\pyy}{-2pt}
    \setlength{\myhi}{\pyy}
    \addtolength{\myhi}{-1\py}
    \par
    \parbox[b]{\textwidth}{%
    \rule[\py]{\RW}{\myhi}%
    \hskip -\RW%
    \rule[\pyy]{\px}{\RW}%
    \hskip -\px%
    \raggedright%
    \CNV\FmN{\@chapapp}\space\CNoV\thechapter%
    \hskip1pt%
    \mghrulefill{\RW}%
    \rule{\RW}{\pyy}\par\nobreak%
    \vskip -\baselineskip%
    \vskip -\pyy%
    \hskip \mylen%
    \mghrulefill{\RW}\par\nobreak%
    \vskip \pyy}%
    \vskip 20\p@}

\renewcommand{\DOTI}[1]{%
\CTV\FmTi{#1}\par\nobreak
\vskip 40\p@}

  \renewcommand{\DOTIS}[1]{%
    \mghrulefill{\RW}\par\nobreak
    \CTV\FmTi{#1}\par\nobreak
    \vskip 60\p@
    }
\makeatother
%%%=============================================================


\newcommand{\ie}{c.-\`a-d.~}
\hbadness=10000% pb d'overfull box 
\hfuzz=50pt

\setcounter{secnumdepth}{3}
\setcounter{tocdepth}{3}

\makeindex
\def\underscore{\char`\_}
\pdfoptionpdfminorversion=5

\makeatletter
\renewcommand{\thesection}{\arabic {section}}
\renewcommand{\SC@figure@vpos}{c}% centrer verticalement le caption avec le package sidecap...
\renewcommand{\fnum@figure}{\small\textbf{Figure~\thefigure}}
\renewcommand{\fnum@table}{\small\textbf{Tableau~\thetable}}
\newcommand\figcaption{\def\@captype{figure}\caption}
\newcommand\tabcaption{\def\@captype{table}\caption}
\makeatother

\def\thechapter{\Roman{chapter}}

%%%=============================================================

%-----<START>------

\begin{document}

%=============================================================
% Pictures Folders Definition 
%=============================================================
\graphicspath{{chapter1/fig/}{chapter2/fig/}{chapter3/fig/}}  % Figures input

%%%=============================================================
% Information
%%%============================================================= 
\Titre{Contrôle de trafic et gestion\\ de la QoS "IP centric"\\ dans les réseaux\\mobiles LTE}
\Titreen{QoS management and traffic control in \\LTE mobile networks based on "IP centric" approach}
\DateSoutenance{xx xxxxxxxx 2016}
\DateDefence{xxxxxxxx xx, 2016}
\Auteur{William David}{Diego Maza}
\President{M}{Dr}{Prenom}{Nom}{Titre, entité}{Title, entity}
\Rapporteur{M}{Dr}{Prenom}{Nom}{Titre, entité}{Title, entity}{City, Country2}
\Rapporteur{M}{Dr}{Prenom}{Nom}{Titre, entité}{Title, entity}{City, Country3}
\Examinateur{M}{Dr}{Prenom}{Nom}{Titre, entité}{Title, entity}{City, Country4}
\Examinateur{M}{Dr}{Prenom}{Nom}{Titre, entité}{Title, entity}{City, Country5}
\Directeur{M}{Prof}{Xavier}{Lagrange}{Professeur, Télécom Bretagne}{Full Professor, Télécom Bretagne}{Directeur de thèse}

\Encadrant{Mme}{Dr}{Isabelle}{Hamchaoui}{Ingénieur de Recherche Senior, Orange Labs}{Senior Research Engineer, Orange Labs}{City, Country}

\Resumes{Titre}{The thesis abstract is the first thing that your examiner reads. It sets the tone of what is to come. On the basis of the abstract alone, before they start the text proper, the examiner will form some expectations about what is in store – how well the thesis is likely to be written, whether it is going to be well argued and evidenced, whether it is going to be lively or dull. While the abstract is a short piece of writing, it is a very important little text.}{The thesis abstract is the first thing that your examiner reads. It sets the tone of what is to come. On the basis of the abstract alone, before they start the text proper, the examiner will form some expectations about what is in store – how well the thesis is likely to be written, whether it is going to be well argued and evidenced, whether it is going to be lively or dull. While the abstract is a short piece of writing, it is a very important little text.}

\Composante{Télécom Bretagne}{Technopôle Brest-Iroise - CS 83818 - 29238 Brest Cedex 3\\ Tél : + 33(0) 29 00 11 11 - Fax : + 33(0) 29 00 10 00}{En accréditation conjointe avec l'Ecole doctorale Matisse}{préparée dans le département Réseaux, Sécurité et Multimédia (RSM) \\ Laboratoire Irisa}

%%%=============================================================
% Front Cove
%%%============================================================= 
\addtocounter{page}{-1}
\makeFrontCove
\whitepage
\newpage\thispagestyle{empty}\addtocounter{page}{-1}
~\newpage\thispagestyle{empty}\addtocounter{page}{-1}
%%%=============================================================
% Data in French
%%%============================================================= 
\cover
\whitepage
\newpage\thispagestyle{empty}\addtocounter{page}{-1}
~\newpage\thispagestyle{empty}\addtocounter{page}{-1}
%%%=============================================================
% Data in English
%%%============================================================= 
\coveren
\whitepage
\newpage\thispagestyle{empty}\addtocounter{page}{-1}
~\newpage\thispagestyle{empty}\addtocounter{page}{-1}
%%%=============================================================
% Acknowledgements
%%%============================================================= 

\frontmatter %turns off chapter numbering and uses roman numerals
\cleardoublepage
\phantomsection
\chapter*{Acknowledgements}
\addcontentsline{toc}{chapter}{Acknowledgements}

%===================================================================
\pagestyle{fancy}
\fancyhf{}
\fancyhead[LE,RO]{\itshape Acknowledgements}
\fancyfoot[RO]{\thepage}
\fancyfoot[LE]{\thepage}
\renewcommand{\headrulewidth}{0.5pt}
\renewcommand{\footrulewidth}{0pt}
%===================================================================


\blindtext


%%%=============================================================
% Abstract
%%%============================================================= 

\cleardoublepage
\phantomsection
\chapter*{Abstract}
\addcontentsline{toc}{chapter}{Abstract}

%===================================================================
\pagestyle{fancy}
\fancyhf{}
\fancyhead[LE,RO]{\itshape Abstract}
\fancyfoot[RO]{\thepage}
\fancyfoot[LE]{\thepage}
\renewcommand{\headrulewidth}{0.5pt}
\renewcommand{\footrulewidth}{0pt}
%===================================================================

\Blindtext

%%%=============================================================
% Acronyms
%%%============================================================= 

\cleardoublepage
\phantomsection
\chapter*{Acronyms}
\addcontentsline{toc}{chapter}{List of Acronyms}
%===================================================================
\pagestyle{fancy}
\fancyhf{}
\fancyhead[LE,RO]{\itshape List of Acronyms}
\fancyfoot[RO]{\thepage}
\fancyfoot[LE]{\thepage}
\renewcommand{\headrulewidth}{0.5pt}
\renewcommand{\footrulewidth}{0pt}
%===================================================================

\begin{acronym}[Sched]

\acro{TTI}{Time Transmission Interval}
\acro{RB}{Resource Block}
\acro{PRB}{Physical Resource Block}
\acro{VRB}{Virtual Resource Block}
\acro{UE}{User Equipment}
\acro{RBG}{Resource Block Group}
\acro{LA}{Link Adaptation}
\acro{AMC}{Adaptive Modulation and Coding}
\acro{MCS}{Modulation and Coding Scheme}
\acro{CQI}{Channel Quality Indicator}
\acro{BLER}{Block Error Rate}
\acro{QoS}{Quality of Service}
\acro{IP}{Internet Protocol}
\acro{LTE}{Long-Term Evolution}
\acro{DiffServ}{Differentiated Services}
\acro{E2E}{End-to-End}
\acro{eNB}{evolved Node B}
\acro{PHB}{Per-Hop Behavior}
\acro{PCC}{Policy and Charging Control}
\acro{LENA}{LTE-EPC Network Simulator}
\acro{KPI}{Key Performance Indicators}
\acro{3GPP}{3rd Generation Partnership Project}
\acro{RLC}{Radio Link Control}
\acro{MAC}{Medium Access Control}
\acro{PDCP}{Packet Data Convergence Protocol}
\acro{P-GW}{Packet Data Network Gateway}
\acro{TEID}{Tunnel Endpoint ID}
\acro{ITU}{International Telecommunication Union}
\acro{ETSI}{European Telecommunications Standards Institute}
\acro{IETF}{Internet Engineering Task Force}
\acro{S-GW}{Serving Gateway}

\acro{CAPEX}{CApital EXpenditure}
\acro{OPEX}{Operational EXpenditure}
\acro{IntServ}{Integrated Services}

\acro{TM}{Transparent Mode}
\acro{UM}{Unacknowledged Mode}
\acro{AM}{Acknowledged Mode}
\acro{ARQ}{Automatic Repeat reQuest}
\acro{HARQ}{Hybrid Automatic Repeat reQuest}
\acro{EPS}{Evolved Packet System}
\acro{DSCP}{Differentiated Services Code Point}
\acro{EUTRAN}{Evolved Universal Terrestrial Radio Access Network}
\acro{EPC}{Evolved Packet Core}

\acro{QCI}{QoS Class Identifier}
\acro{ARP}{Allocation and Retention Priority}
\acro{GBR}{Guaranteed Bit Rate}
\acro{MBR}{Maximum Bit Rate}
\acro{PDP}{Packet Data Protocol}

\acro{CS}{Circuit Switched}
\acro{PS}{Packet Switched}
\acro{OSI}{Open Systems Interconnection}

\acro{OTT}{Over-The-Top}
\acro{ISP}{Internet Service Provider}

\acro{DASH}{Dynamic Adaptive Streaming over HTTP}
\acro{NP}{Network Performance}
\acro{MME}{Mobility Management Entity}

\acro{UMTS}{Universal Mobile Telecommunications System}
\acro{HSDPA}{High Speed Data Packet Access}
\acro{ATM}{Asynchronous Transfer Mode}
\acro{RSVP}{Resource Reservation Protocol}
\acro{DSCP}{Differentiated Services Code Point}
\acro{FIFO}{First In, First Out}
\acro{TC}{Traffic  Class}
\acro{ToS}{Type of Service}
\acro{SLA}{Service Level Agreements}
 
\acro{FTTH}{Fiber to the Home}

\acro{IMSI}{International Mobile Subscriber Identity}
\acro{GTP}{GPRS Tunneling Protocol}

\acro{RNC}{Radio Network Controller}
\acro{GPRS}{General Packet Radio Service}
\acro{SGSN}{Serving GPRS Support Node}
\acro{GGSN}{Gateway GPRS Support Node}
\acro{APN}{Access Point Name}
\acro{PDN}{Packet Data Network}
\acro{EPS}{Evolved Packet System}
\acro{SAE}{System Architecture Evolution}

\acro{E-UTRAN}{Evolved Universal Terrestrial Access Network}
\acro{MIMO}{Multiple-Input-Multiple-Output}
\acro{OFDMA}{Orthogonal Frequency Division Multiple Access}
\acro{QoE}{Quality of Experience}

\acro{TFT}{Traffic Flow Template}

\acro{NAS}{Non Access Stratum}

\acro{RRM}{Radio resource management}

\acro{EMM}{EPS Mobility Management}
\acro{ESM}{Session Management}

\acro{RRC}{Radio Resource Control}

\acro{DRB}{Data Radio Bearer}
\acro{PHY}{Physical Layer}

\acro{MBR}{Maximum Bit Rate}
\acro{AMBR}{Aggregated Maximum Bit Rate}
\acro{TEID}{Tunnel Endpoint ID}

\acro{SDF}{Service Data Flow}
\acro{HSS}{Home Subscriber Server}


\acro{E-RAB}{E-UTRAN Radio Access Bearer}

\acro{RAB}{Radio Access Bearer}

\acro{ROHC}{Robust Header Compression}

\acro{SDU}{Service Data Unit}
\acro{PDU}{Packet Data Unit}
\acro{PCI}{Protocol Control Information}

\acro{ARQ}{Automatic Repeat reQuest}

\acro{PCEF}{Policy and Charging Enforcement Function}

\acro{BBERF}{Bearer Binding and Event Reporting Function}

\acro{PCRF}{Policy and Charging Rules Function}

\acro{AF}{Application Function}

\acro{OCS}{Offline Online Charging System}

\acro{SPR}{Subscription Profile Repository}

\acro{SDF}{Service Data Flow}

\acro{SMS}{Short Message Service}

\acro{HAS}{HTTP Adaptive Streaming}

\acro{SINR}{Signal to Interference plus Noise Ratio}

\end{acronym}


%%%=============================================================
% Contents
%%%============================================================= 
\cleardoublepage
\pagestyle{fancy}
\fancyhf{}
\fancyhead[LE,RO]{Contents}
\fancyfoot[LE]{\thepage}
\fancyfoot[RO]{\thepage}
\renewcommand{\headrulewidth}{0.5pt}
\renewcommand{\footrulewidth}{0pt}
\phantomsection
\tableofcontents
\addcontentsline{toc}{chapter}{Contents} 

%%%=============================================================
% Chapters
%%%============================================================= 

\mainmatter %turns on chapter numbering, resets page numbering and uses arabic numerals for page numbers


%}----> Put Chapters here:
\cleardoublepage
\phantomsection
\chapter*{Introduction}
\addcontentsline{toc}{chapter}{Introduction}
%==============================================================================
\pagestyle{fancy}
\fancyhf{}
\fancyhead[LE,RO]{\itshape Introduction}
\fancyfoot[RO]{\thepage}
\fancyfoot[LE]{\thepage}
\renewcommand{\headrulewidth}{0.5pt}
\renewcommand{\footrulewidth}{0pt}
%==============================================================================

\Blindtext

\cleardoublepage %%%
\chapter{Chapter One}
%===================================================================
\pagestyle{fancy}
\fancyhf{}
\fancyhead[RO]{\itshape \rightmark} %
\fancyhead[LE]{\itshape \leftmark}
\fancyfoot[RO]{\thepage}
\fancyfoot[LE]{\thepage}
\renewcommand{\headrulewidth}{0.5pt}
\renewcommand{\footrulewidth}{0pt}
%===================================================================

\section{Section 1}
\blindmathpaper
\section{Section 2}
\blindmathpaper
\section{Section 3}
\blindmathpaper
\section{Section 4}
\blindmathpaper
\cleardoublepage %%%
\chapter{Chapter Two}
%===================================================================

\section{Section 1}
\blindmathpaper
\section{Section 2}
\blindmathpaper
\section{Section 3}
\blindmathpaper
\section{Section 4}
\blindmathpaper
\cleardoublepage %%%
\chapter{Chapter Three}
%===================================================================


\section{Section 1}
\blindmathpaper
\section{Section 2}
\blindmathpaper
\section{Section 3}
\blindmathpaper
\section{Section 4}
\blindmathpaper
% ...
\cleardoublepage
\phantomsection
\chapter*{Conclusion}
\addcontentsline{toc}{chapter}{Conclusion}
%==============================================================================
\pagestyle{fancy}
\fancyhf{}
\fancyhead[LE,RO]{\itshape Conclusion}
\fancyfoot[RO]{\thepage}
\fancyfoot[LE]{\thepage}
\renewcommand{\headrulewidth}{0.5pt}
\renewcommand{\footrulewidth}{0pt}
%==============================================================================


\blindtext


%%%=============================================================
% Bibliography
%%%============================================================= 
\newpage
\cleardoublepage %cleardoublepage
\bibliographystyle{unsrt}
\bibliography{allBiblio}
\phantomsection

%%%=============================================================
%  Appendix
%%%=============================================================

\begin{appendices}

%}----> Put Appendix here:
%===================================================================
\pagestyle{fancy}
\renewcommand{\chaptermark}[1]{\markboth{\MakeUppercase{Appendix~\thechapter. #1 }}{}}
\renewcommand{\sectionmark}[1]{\markright{\thechapter.\thesection~ #1}}
\fancyhf{}
\fancyhead[RO]{\itshape\rightmark}
\fancyhead[LE]{\itshape\leftmark}
\fancyfoot[RO]{\thepage}
\fancyfoot[LE]{\thepage}
\renewcommand{\headrulewidth}{0.5pt}
\renewcommand{\footrulewidth}{0pt}

%===================================================================
\cleardoublepage
\chapter{Example One}
\phantomsection
%===================================================================

\section{Section 1}
\blindmathpaper
\section{Section 2}
\blindmathpaper
\section{Section 3}
\blindmathpaper
\section{Section 4}
\blindmathpaper
%===================================================================
\cleardoublepage
\chapter{Example Two}
\phantomsection
%===================================================================


\section{Section 1}
\blindmathpaper
\section{Section 2}
\blindmathpaper
\section{Section 3}
\blindmathpaper
\section{Section 4}
\blindmathpaper
%}---->

\end{appendices}
%%%=============================================================
%  List of Figures
%%%============================================================= 

\cleardoublepage
\phantomsection
\listoffigures
\addcontentsline{toc}{chapter}{\listfigurename}

%%%=============================================================
%  List of Tables
%%%============================================================= 

\cleardoublepage
\phantomsection
\listoftables
\addcontentsline{toc}{chapter}{\listtablename}

%%%=============================================================
%  Abstracts End
%%%============================================================= 

\cleardoublepage
\newpage\thispagestyle{empty}\addtocounter{page}{1}
\resumes

%%%============================================================= 
\end{document}


